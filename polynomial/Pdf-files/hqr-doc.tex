\documentclass[12pt,a4paper]{article}



\usepackage{amssymb}

\usepackage{graphicx}

\usepackage{amsmath}



\textheight235mm

\textwidth160mm

\voffset-20mm

\hoffset-10mm

\parindent0cm

\parskip5mm



\begin{document}



{\Huge \textbf{hqr}}

\bigskip

\bigskip



{\Large \textbf{Purpose}}



Hyperbolic QR factorization

\bigskip



{\Large \textbf{Syntax}}



{\tt [Q,R,Z] = hqr(A,J)



[Q,R,Z] = hqr(A,J,tol)}

\bigskip



{\Large \textbf{Description}}



Given a square matrix $A$ and a diagonal signature matrix $J$, the

function computes a hyperbolic QR factorization of $A$ such that

matrix

\[

QA = R

\]

has as many zero rows as the rank defect of $A$, and matrix $Q$ is

$J$-orthogonal, i.e.

\[

Q^T J Q = J.

\]

Output vector {\tt Z} contains indices of non-zero rows in matrix

$R$.



The above factorization exists if and only if

\[

\mathrm{rank}\:A = \mathrm{rank}\: A^T J A

\]

otherwise the function returns empty output arguments.



An optional tolerance parameter {\tt tol} can be specified as an

input argument. It is used as an absolute threshold to detect zero

elements during the reduction process. By default it is the global

zeroing tolerance.



\bigskip



{\Large \textbf{Examples}}



Consider first a standard example. One can check that the obtained

matrix {\tt Q} is not orthogonal.

\begin{verbatim}

>> A

A =

     1     1     1     0

     0     0     1     0

     0     0     1     0

     0     0     1     0



>> J

J =



    -1     0     0     0

     0     1     0     0

     0     0    -1     0

     0     0     0     1



>> [Q,R,Z] = hqr(A,J)

Q =

    1.0000         0         0         0

         0    1.0000   -1.0000    1.0000

         0   -0.7071    1.4142   -0.7071

         0   -0.7071         0    0.7071

R =

    1.0000    1.0000    1.0000         0

         0         0    1.0000         0

         0         0   -0.0000         0

         0         0         0         0

Z =

     1     2



>> svd(Q)

ans =

    2.4142

    1.0000

    1.0000

    0.4142

\end{verbatim}



This second example shows that the location of zero rows in $R$

is not arbitrary.

\begin{verbatim}

>> A

A =

     1     0     1     8

     0     5     0     0

     1     2     1     8

     7     8     7     8



>> J 

J =

     1     0     0     0

     0     1     0     0

     0     0    -1     0

     0     0     0    -1



>> [Q,R,Z] = hqr(A,J)

Q =

    2.5000    1.0000   -2.5000   -0.0000

   -0.2337    0.9872   -0.1612    0.0564

   -2.2987   -0.9872    2.6936   -0.0564

   -0.1429         0    0.1429    1.0000

R =

   -0.0000         0   -0.0000    0.0000

   -0.0000    5.0649   -0.0000   -2.7077

    0.0000         0    0.0000    2.7077

    7.0000    8.2857    7.0000    8.0000

Z =

     4     2     3

\end{verbatim}



Finally, this example shows that when the rank assumption on input

matrices {\tt A} and {\tt J} is not satisfied, then function {\tt hqr}

returns empty output arguments.

\begin{verbatim}

>> A

A =

     1     0

     1     0



>> J

J =

     1     0

     0    -1



>> rank(A'*J*A)

ans =

     0



>> [Q,R,Z]=hqr(A,J)

Q =

     []

R =

     []

Z =

     []

\end{verbatim}



\bigskip



{\Large \textbf{Algorithm}}



The algorithm is described in reference [1]. It consists in

zeroing iteratively entries in each column of $A$ by application

of orthogonal (plane) rotations, row permutations and

$J$-orthogonal (hyperbolic) rotations. Note that, unlike

orthogonal rotations, $J$-orthogonal rotations may be potentially

numerically instable.



\bigskip



{\Large \textbf{References}}



[1] D. Henrion, P. Hippe. Hyperbolic QR factorization for J-spectral

factorization of polynomial matrices. LAAS-CNRS Research Report,

Toulouse, France, February 2003



\bigskip



{\Large \textbf{Diagnostics}}



The function issues error messages under the following conditions:

\begin{itemize}

\item Input matrix {\tt A} is not square.

\item Input matrix {\tt J} has a zero diagonal entry.

\end{itemize}



\bigskip



{\Large \textbf{See also}}



{\tt qr} \hspace{1em} Standard QR factorization.



\end{document}

